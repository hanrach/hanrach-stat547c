%%%%%%%%%%%%%%%%%%%%%%%%%%%%%%%%%%%%%%%%%%%%%%%%%%%%%%%%%%%%%%%%%%%%%%%%%%%%%%%%%%%%
% Template for STAT 547C Final Project Outline
% Author: Rachel Han <hanrach@math.ubc.ca>
% Date: Oct. 26, 2019
% Acknowledgments: ETH, Peter Orbanz, John Cunningham
%%%%%%%%%%%%%%%%%%%%%%%%%%%%%%%%%%%%%%%%%%%%%%%%%%%%%%%%%%%%%%%%%%%%%%%%%%%%%%%%%%%%

\documentclass[]{STAT_547C}
\usepackage{STAT_547C}
% NOTE: change the name and email address to your name in STAT_547C.sty
\usepackage{hyperref}
\hypersetup{colorlinks,
	citecolor=blue
}
\usepackage{booktabs}
\usepackage{amsmath,amsthm,amssymb,amsfonts}
\usepackage[sorting=none,backend=biber,bibstyle=alphabetic,citestyle=alphabetic,giveninits=true,natbib=true]{biblatex}
\addbibresource{../../ref/STAT_547C.bib}
%\bibliography{../../ref/STAT_547C.bib} % add the title and location of your bibliography file

\begin{document}

% NOTE: You will replace the title below with your actual Title.
\makeGenericHeader{Nonlinear state estimation \\ and its Application to Lithium-ion Battery Modelling}{Project Outline}
\vspace{-2cm}


%%%%%%%%%%%%%%%%%%%
\section{Title}

The working title of my project is \emph{Nonlinear state estimation and its Application to Lithium-ion Battery Modelling}.  

%%%%%%%%%%%%%%%%%%%
\section{Background}

Nonlinear state estimation refers to estimating the true \textit{state} of a dynamic model where the underlying relationship between the \textit{observations} (which are often corrupted versions of the \textit{state}) and the \textit{state} is nonlinear. There are two methods of nonlinear state estimation which I am interested in: the Unscented Kalman filter and the particle filter. Both techniques allow us to estimate the state iteratively and importantly, in real time. Nonlinear state estimation has many applications in areas such as process control, robotics and in particular, Lithium-ion battery management systems. The goal of this project is to understand the probability behind the methods, such as conditional probability, Gauss-Markov models, sequential Monte-Carlo methods and resampling methods. Then, I will study ways to predict the state of charge of a Lithium-ion battery using these methods. In particular, I will review these literature on this topic, \cite{Tulsyan:2016} and \cite{Sun:2011}.
%%%%%%%%%%%%%%%%%%%
\section{Technical aspects}

The project will explore the following technical aspects: Gaussian measures, best linear unbiased estimator (BLUE), Gauss-Markov models, conditonal densities and sequential Monte-Carlo method.

%%%%%%%%%%%%%%%%%%%
\section{Literature}

The key references for this project are:

\begin{itemize}
  \item \cite{Patwardhan:2012} contains a recent review on nonlinear state estimation.
  \item \cite{Sullivan:2015} is a technical introduction to Kalman filter. \cite{Maybeck:1982} and \cite{Welch:Bishop:2001} are also references for Kalman filter.
  \item \cite{Tulsyan:2016:ParticleFilter} is a short handbook guide to particle filter.
  \item \cite{Gyorgy:2014} compares Unscented Kalman filter and the particle filter.
  \item \cite{Sun:2011} is an application of Unscented Kalman filter to Lithium-ion state of charge estimation.
  \item \cite{Tulsyan:2016} is an application of a version of particle filter to Lithium-ion state of charge estimation.
\end{itemize}


%%%%%%%%%%%%%%%%%%%
\section{Plan}

I will carry out this project with the following sequence of steps: 
\begin{enumerate}
  
  \item I will study the theory of linear Kalman filter which includes Gaussian measures, BLUE and Gauss-Markov models. Then I can see how the Unscented Kalman filter method extends from the linear case. For particle filtering, I will study the sequential Monte Carlos method.
  
  \item I will compare how the two methods are different in terms of theory and computational effort. I will refer to \cite{Gyorgy:2014}. 

  \item I will focus on one of the methods and apply the method on a simple example to demonstrate how it works.
  
  \item I will focus on only one of \cite{Tulsyan:2016} and \cite{Sun:2011}. Then I will give a brief overview of the battery model presented and how we can apply the selected filtering method. I will summarise the results of the selected paper. Difficulty of the approach will also be discussed.
\end{enumerate}


%%%%%%%%%%%%%%%%%%%
\section{Why I'm interested in this topic}

I am interested in applying probability to model or predict physical systems. My background is in building a fast Lithium ion battery simulator comprised of partial differential equations (PDE). I think it will be interesting to use probability with the PDE model to predict a useful feature of the battery. 

\printbibliography

\end{document}

